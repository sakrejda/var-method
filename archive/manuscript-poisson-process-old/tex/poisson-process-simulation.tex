
\chapter{Count data simulation for a continuous-time dynamics.}

For a simulation where the timing of events wihin batches is not
important but the number of events in a batch is important, it is
possible to avoid dealing with timing entirely and simply simulate
from the Poisson process conditional on $m(s,t)$ for each batch.
The key point for such a simulation becomes the calculation of $m(s,t)$.

To simulate event timing a thinning (rejection) algorithm is available which and
its efficiency depends on how well the proposed event distribution
matches the function $\lambda(t)$ as used for simulation or estimation.

\section{Splines for generating a smooth function}

To evaluate models constructed based on this Poisson process formulation
it will be important to have a way of simulating from the Poisson
process as described above and evaluating our ability to recover
parameters.  The appropriate splines and their integrals can be
constructed from a small set of R functions which implement the
calculations described in section \ref{sec:ihpp-implementation}.

\begin{knitrout}
\definecolor{shadecolor}{rgb}{0.969, 0.969, 0.969}\color{fgcolor}\begin{kframe}
\begin{alltt}
\hlkwd{read_chunk}\hlstd{(}\hlstr{'~/packages/cruftery/package_dir/R/spline_functions.R'}\hlstd{)}
\end{alltt}
\end{kframe}
\end{knitrout}

For any spline the function value at point x can be described as a
function of the basis function.  The rest of the calculations are
generic.

\begin{knitrout}
\definecolor{shadecolor}{rgb}{0.969, 0.969, 0.969}\color{fgcolor}\begin{kframe}
\begin{alltt}
\hlcom{# Generic radial spline:}
\hlstd{radial_spline} \hlkwb{<-} \hlkwa{function}\hlstd{(}\hlkwc{x}\hlstd{,} \hlkwc{f}\hlstd{,} \hlkwc{knot_points}\hlstd{,} \hlkwc{knot_weights}\hlstd{,} \hlkwc{knot_scale}\hlstd{) \{}
  \hlkwa{if} \hlstd{(}\hlkwd{length}\hlstd{(x)} \hlopt{==} \hlnum{1}\hlstd{) \{}
    \hlstd{knot_contributions} \hlkwb{<-} \hlkwd{f}\hlstd{(}\hlkwc{x}\hlstd{=x,} \hlkwc{center}\hlstd{=knot_points,} \hlkwc{scale}\hlstd{=knot_scale)}\hlopt{*}\hlstd{knot_weights}
    \hlkwd{return}\hlstd{(}\hlkwd{sum}\hlstd{(knot_contributions))}
  \hlstd{\}} \hlkwa{else} \hlstd{\{}
    \hlkwd{return}\hlstd{(}\hlkwd{sapply}\hlstd{(x,radial_spline,} \hlkwc{f}\hlstd{=f,} \hlkwc{knot_points}\hlstd{=knot_points,} \hlkwc{knot_weights}\hlstd{=knot_weights,} \hlkwc{knot_scale}\hlstd{=knot_scale))}
  \hlstd{\}}
\hlstd{\}}
\end{alltt}
\end{kframe}
\end{knitrout}

The generic spline function can be specialized for a gaussian spline
as follows.
\begin{knitrout}
\definecolor{shadecolor}{rgb}{0.969, 0.969, 0.969}\color{fgcolor}\begin{kframe}
\begin{alltt}
\hlcom{# Generic radial spline specialized to Gaussian.}
\hlstd{gaussian_spline} \hlkwb{<-} \hlkwa{function}\hlstd{(}\hlkwc{x}\hlstd{,} \hlkwc{knot_points}\hlstd{,} \hlkwc{knot_weights}\hlstd{,} \hlkwc{knot_scale}\hlstd{) \{}
  \hlstd{f} \hlkwb{<-} \hlkwa{function}\hlstd{(}\hlkwc{x}\hlstd{,} \hlkwc{center}\hlstd{,} \hlkwc{scale}\hlstd{)} \hlkwd{dnorm}\hlstd{(}\hlkwc{x}\hlstd{=x,} \hlkwc{mean}\hlstd{=center,} \hlkwc{sd}\hlstd{=scale)}
  \hlkwd{return}\hlstd{(}\hlkwd{radial_spline}\hlstd{(}\hlkwc{x}\hlstd{=x,} \hlkwc{f}\hlstd{=f,} \hlkwc{knot_points}\hlstd{=knot_points,} \hlkwc{knot_weights}\hlstd{=knot_weights,} \hlkwc{knot_scale}\hlstd{=knot_scale))}
\hlstd{\}}
\end{alltt}
\end{kframe}
\end{knitrout}

The integral can be coded in terms of R's CDF functions.
\begin{knitrout}
\definecolor{shadecolor}{rgb}{0.969, 0.969, 0.969}\color{fgcolor}\begin{kframe}
\begin{alltt}
\hlcom{# Integral of Gaussian radial spline.}
\hlstd{gaussian_spline_integral} \hlkwb{<-} \hlkwa{function}\hlstd{(}\hlkwc{knot_points}\hlstd{,} \hlkwc{knot_weights}\hlstd{,} \hlkwc{knot_scale}\hlstd{,} \hlkwc{a}\hlstd{,} \hlkwc{b}\hlstd{) \{}
  \hlstd{contributions} \hlkwb{<-} \hlkwd{mapply}\hlstd{(}
    \hlkwc{FUN}\hlstd{=}\hlkwa{function}\hlstd{(}\hlkwc{point}\hlstd{,} \hlkwc{weight}\hlstd{,} \hlkwc{scale}\hlstd{,} \hlkwc{a}\hlstd{,} \hlkwc{b}\hlstd{) \{}
      \hlstd{weight} \hlopt{*} \hlstd{(}\hlkwd{pnorm}\hlstd{(}\hlkwc{q}\hlstd{=b,} \hlkwc{mean}\hlstd{=point,} \hlkwc{sd}\hlstd{=scale)} \hlopt{-} \hlkwd{pnorm}\hlstd{(}\hlkwc{q}\hlstd{=a,} \hlkwc{mean}\hlstd{=point,} \hlkwc{sd}\hlstd{=scale))}
    \hlstd{\},}
    \hlkwc{point} \hlstd{= knot_points,} \hlkwc{weight} \hlstd{= knot_weights,}
    \hlkwc{MoreArgs} \hlstd{=} \hlkwd{list}\hlstd{(}\hlkwc{scale}\hlstd{=knot_scale,} \hlkwc{a}\hlstd{=a,} \hlkwc{b}\hlstd{=b)}
  \hlstd{)}
  \hlkwd{return}\hlstd{(}\hlkwd{sum}\hlstd{(contributions))}
\hlstd{\}}
\end{alltt}
\end{kframe}
\end{knitrout}

Given the spline integral function, it is simple to calculate $m(s,t)$
and simulate the number of events in a batch as:

\begin{align}
X \sim Poi(m(s,t)) 
\end{align}










