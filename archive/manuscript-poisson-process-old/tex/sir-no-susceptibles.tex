\chapter{The SIR model can be restated to ignore susceptibles.}\label{chap:sir-nosusceptibles}
%\epigraph{}{ }

The SI model is a simplification of the SIR model for a disease where infectious individuals leave
the infectious pool after one time step.  

\section{SI case}
The SI model  can be stated as a pair of equations describing
the size of the susceptible (S) and infected (I) population at time $t$
in terms of the system state at time $t-1$ and three groups of parameters: 
1) $r_t$, which describes the infectiousness of the disease at time $t$;
2) $\alpha_1$, and $\alpha_2$ which describe how effectively the
infected interact with the susceptible population and how effectively
the susceptible population mixes with the infected population; and 3)
the error terms $\epsilon_t$ and $u_t$.  The model is stated in equation
\ref{eq:sir-basic}.

\begin{align}\label{eq:sir-basic}
I_t &= r_t I_{t-1}^{\alpha_1} S_{t-1}^{\alpha_2} \epsilon_{t} \\
S_t &= B_{t-d} + S_{t-1} - I_t + u_t
\end{align}

One of the difficulties of applying this model is that the state
variables are not observed directly.  Only a case count (C) is observed
which is related to the number infected at time $t$ through a reporting
fraction $\rho_t$ as $I_t = \rho_t C_t$.  For estimation and to
understand the confounding which results from these limited observations
it would be useful to state the model in terms of infected individuals
only, leaving aside the dynamics of the susceptible population.
According to CITE CITE CITE this can be accomplished.  First we begin
by shifting the susceptible equation one time step back and using it to 
replace the $S_{t-1}$ term in the infectious equation:

\begin{align}
I_t &= r_t I_{t-1}^{\alpha_1} 
	\left[
		B_{t-1-d} + S_{t-2} - I_{t-1} + u_{t-1}
	\right]^{\alpha_2} \epsilon_{t} \\
\end{align}

Now the only term left for susceptibles is $S_{t-2}$ which we can
replace by shifting the infectious equation back one time step and
isolating $S$ on one side.

\begin{align}
S_{t-1}^{\alpha_2} &= \frac{I_t}{r_t I_{t-1}^{\alpha_1} \epsilon_{t}} \\
\alpha_2 \log S_{t-1} &= \log I_t - \log r_t - \alpha_1 \log I_{t-1} - \log \epsilon_{t}  \\
\log S_{t-1} &= \frac{\log I_t - \log r_t - \alpha_1 \log I_{t-1} - \log \epsilon_{t}}{\alpha_2}  \\
\log S_{t-2} &= \frac{\log I_{t-1} - \log r_{t-1} - \alpha_1 \log I_{t-2} - \log \epsilon_{t-1}}{\alpha_2}
\end{align}

This new representation of $S$ refers only to lagged terms of $I$ and
parameters.  It can be exponentiated and used to replace $S_{t-2}$ in the infectious
equation.  

\begin{align}
I_t &= r_t I_{t-1}^{\alpha_1} 
	\left[
		B_{t-1-d} + \exp\left(
			\frac{\log I_{t-1} - \log r_{t-1} - \alpha_1 \log I_{t-2} - \log \epsilon_{t-1}}{\alpha_2}
		\right) - I_{t-1} + u_{t-1}
	\right]^{\alpha_2} \epsilon_{t} \\
\end{align}

Using the log of this equation can clarify the role of different terms:

\begin{align}
\log I_t &= \log r_t + \alpha_1 \log I_{t-1} + \\
				 &	\alpha_2 \log \left[
		B_{t-1-d} + \exp\left(
			\frac{\log I_{t-1} - \log r_{t-1} - \alpha_1 \log I_{t-2} - \log \epsilon_{t-1}}{\alpha_2}
		\right) - I_{t-1} + u_{t-1}
	\right] + \log \epsilon_{t}
\end{align}

The multiplicative error terms appear in log form everywhere so their
expectation will be zero.  In the exponentiated term the (log) ratio of
infected individuals in two previous time steps appears.  Though the
equation is stated in terms of infected counts it could equally be
stated using the reporting equation and case counts rather than true
counts. 

\section{Multi-strain non-interacting SI}
A similar transformation is also possible for a multi-strain SI model
with non-interacting strains.  In this case we need to keep track of
strain-specific susceptibility and infection, which we do by
subscripting with one bit per strain.  An individual susceptible to all
strains is denoted as $S_{0000}$ and an individual immune to all but
the third strain is denoted as $S_{1101}$.  An individual susceptible to
to the third strain but with unknown status with regards to all others
is denoted as $S_{xx0x}$.  This last notation refers to a sum:

\begin{align}
	S_{(0xxx),t} &= S_{(0000),t} + S_{(0100),t} + S_{(0010),t} + \\
							 &+ S_{(0001),t} + S_{(0110),t} + S_{(0011),t} + \\
							 &+ S_{(0101),t} + S_{(0111),t}
\end{align}

With this notation in hand we can write the dynamics equations following
the example of the SI model.  

\begin{align}\label{eq:sir-multistrain}
	I_{(1xxx),t} &= r_{1,t} I_{(1xxx),t-1}^{\alpha_1} S_{(0xxx),t-1}^{\alpha_2} \epsilon_{t} \\
	S_{(0xxx),t} &= B_{t-d} + S_{(0xxx),t-1} - I_{(1xxx),t} + u_t
\end{align}

As before, $I_t$ is observed---at least in serotype-specfic data---and
the goal is to substitute out $S_t$ in all equations.  Immediately we
notice that indexing by strain susceptibility in the case of
non-interacting strains does not affect equation structure and we can
recycle the solution from original single-strain case.

\begin{align}
	\log I_{(1xxx),t} &= \log r_{1,t} + \alpha_1 \log I_{(1xxx),t-1} + \\
				 &	\alpha_2 \log \left[
		B_{t-1-d} + \exp\left(
			\frac{\log I_{(1xxx),t-1} - \log r_{1,t-1} - \alpha_1 \log
			I_{(1xxx),t-2} - \log \epsilon_{t-1}}{\alpha_2}
		\right) - I_{(1xxx),t-1} + u_{t-1}
	\right] + \log \epsilon_{t}
\end{align}

Ultimately the change is trivial because there are no interactions and
we are monitoring an important state variable.  

\section{Multi-strain interacting SI}

As a next step, we introduce interactions into the multi-strain SI
model.  In this model the immune history of the individualis allowed to
affect the infection rate.  Due to this change we can no longer lump
all susceptible individuals together in a sum term which complicates the
process of removing susceptibles from the equation.  Conveniently, the
infected individuals still function only as a group regardless of their
previous infection status.

\begin{align}\label{eq:sir-interacting}
I_t &= r_t I_{t-1}^{\alpha_1} S_{t-1}^{\alpha_2} \epsilon_{t} \\
S_t &= B_{t-d} + S_{t-1} - I_t + u_t
\end{align}


