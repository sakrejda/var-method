\begin{abstract}
Motivation:
\begin{enumerate}
\item VAR(p) form a basic class of models used for prediction in multiple
contexts (econometric tradition, ecology).  
\item After fitting, VAR(p) models are often analyzed for stability
(econometric, population biology cites) via a variety of
Eigen-decomposition based numerical methods which produce information on
long-run growth (stationarity/nonstationarity/near-non-stationarity),
oscillations (imaginary eigenvalues), transient dynamics and dampening
or instability).   
\item An alternative view of these diagnostic numbers sees them as
meaningful parameters in a generative model.
\item This alternative view suggests that we could place weak priors on
these parameters and encourage models that produce stable numbers.
\item Such models could still be checked for fit and predictive performance
\item In public health such models could be made to match the expert expectations that
under certain conditions specific diseases either fade or grow
explosively (unless mediated by interventions).  
\item Models tuned based on these priors should be viewed as:
 - descriptive and predictive, encoding an expert-driven bias.
 - in live prediction exercises, comparing biased model predictions to
   data is particularly critical to indicate when a model fails to
   capture unusual features (those that go against regularizing priors).
\end{enumerate}

Methods:
\begin{enumerate}
\item VAR(p) models can be re-written as VAR(1) models with a structured
projection matrix of coefficients, identity sub-matrices, and zero
sub-matrtices.
\item Eigenvalue decomposition of the fitted matrix can be used
diagnostically to indicate whether the fitted model represents a
stationary system, how fast the stationary system will return to
equilibrium after a disturbance, how strong the tendency to oscillate 
is, what periodicity/ies of oscillations is/are important.  These
features are used diagnostically after fitting a VAR(p) model to data.
\item Knowledge of infectious disease dynamics can also be encoded using 
the metrics derived from the components of an Eigenvalue decomposition.
For example infectious disease systems are known to be at least weak
stationary in endemic settings and nearly non-stationary models \emph{can} be
used to model hyperendemic settings.
\item We translate a series of [specific statements] on endemic and
hyperendemic infectious disease time series behavior into priors on
components of the Eigenvalue decomposition of a projection matrix.  
\item We characterize outputs of prior simulations based on reconstructing
the projection matrix from these prior components.
\item We fit VAR(p) models to simulated time series including seasonal, stationary,
and nearly non-stationary series.
\item We fit VAR(p) models to dengue data, we incorporate a reporting delay
model, we compare predictions to SPAMD/prior-unconstrained
SARIMA/frequentist SARIMA/MOA
\end{enumerate}
\end{abstract}
