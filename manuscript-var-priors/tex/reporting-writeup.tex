\section{Reporting process}

The dengue case reporting process is composed of a series of steps, and not all cases pass through all the steps.  In general the clinic or hospital to encounter a patient establishes the date of onset.  The date of admission is also recorded at this step.  Hospitals record cases into a local database or collect paper case records.  Hospitals transmit the case information to a district or provincial health office.  District or provincial offices transmit cases to the ministry of public health.  Some cases do not follow this process smoothly due and arrive at two times: 1) near week 12 there is a dealine for finalizing reporting for the previous calendar year, all cases not reported by the last week of the year are delayed until week 12; and 2) near week 50 the Dengue season is winding down which frees resources for tracking down and sending on ``stray'' cases.  The reporting process is heterogeneous and our understanding of this reporting process is limited. 

The data which arrives at our doorstep has two spatial codes: 1) province; and 2) ``addrcode''.  The province code corresponds to codes given in the ISO 3166-2 standard entry for Thailand.  The ``addrcode'' field is composed of six characters (AABBCC)and each pair designates a nested spatial location: AA: district, BB: subdistrict, CC: village.  These are non-unique (village ``10'' appears in multiple districts) and we re-code them to be unique.  Boundaries of these administrative units have shifted at different times and their hierarchical organization has also shifted which results in inconsistent coding over time especially below the province level.

Data generally arrives smoothly but there is evidence that batches of records are delayed and release all a once into the reporting system.  Additionally, each year data arrives smoothly until December 31st, and then all remaining cases are held until the following April for a final report resulting in an interesting truncation process (see simple simulation for the effect of truncation on reporting intervals).
